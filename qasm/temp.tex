%     qubit q1
%     qubit q2
%     qubit q3
%      S q1
%      cnot q2,q1
%      h q2
%      cnot q2,q1
%      S q1

%  Time 01:
%    Gate 00 S(q1)
%  Time 02:
%    Gate 01 cnot(q2,q1)
%  Time 03:
%    Gate 02 h(q2)
%  Time 04:
%    Gate 03 cnot(q2,q1)
%  Time 05:
%    Gate 04 S(q1)

% Qubit circuit matrix:
%
% q1: gAxA, gBxA, n  , gDxA, gExA, n  
% q2: n  , gBxB, gCxB, gDxB, n  , n  
% q3: n  , n  , n  , n  , n  , n  

\documentclass[11pt]{article}
\input{xyqcirc.tex}

% definitions for the circuit elements

\def\gAxA{\op{P}\w\A{gAxA}}
\def\gBxB{\b\w\A{gBxB}}
\def\gBxA{\o\w\A{gBxA}}
\def\gCxB{\op{H}\w\A{gCxB}}
\def\gDxB{\b\w\A{gDxB}}
\def\gDxA{\o\w\A{gDxA}}
\def\gExA{\op{P}\w\A{gExA}}

% definitions for bit labels and initial states

\def\bA{ \q{q_{1}}}
\def\bB{ \q{q_{2}}}
\def\bC{ \q{q_{3}}}

% The quantum circuit as an xymatrix

\xymatrix@R=5pt@C=10pt{
    \bA & \gAxA &\gBxA &\n   &\gDxA &\gExA &\n  
\\  \bB & \n   &\gBxB &\gCxB &\gDxB &\n   &\n  
\\  \bC & \n   &\n   &\n   &\n   &\n   &\n  
%
% Vertical lines and other post-xymatrix latex
%
\ar@{-}"gBxA";"gBxB"
\ar@{-}"gDxA";"gDxB"
}

\end{document}
